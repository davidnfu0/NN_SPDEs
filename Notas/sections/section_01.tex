\section{Introducción y algunas ecuaciones importantes}

\subsection{Introducción}
El tipo de ecuaciones que se estudiaran en este trabajo son una generalización de las ecuaciones diferencias estocásticas (EDEs), aveces llamadas ecuaciones de Itô. La generalización a espacios de dimensión infinita, es útil y necesaria para, por ejemplo, resolver ecuaciones estocásticas de tipo parabolico, es decir, ecuaciones que involuacran derivadas parciales en tiempo y en espacio, además de una fuente de aleatoriedad. El estudio análitico de este tipo de ecuaciones es complejo y en la gran mayoría de los casos no se pueden encontrar soluciones explícitas, lo que hace necesario recurrir a métodos numéricos para simular sus soluciones.

En este trabajo simularemos la solución de una ecuación estocástica en dimensión infinita, utilizando redes neuronales. Para ello introduciremos formalmente el concepto de ecuación estocástica en dimensión infita y haremos uso de resultados importantes de la teoría de procesos estocásticos y ecuaciones diferenciales estocásticas. Además, presentaremos resultados teóricos de aproximación mediante redes neuronales y explicitaremos un algoritmo que permita construir una red neuronal para simular la solución de una ecuación dada. Para finalizar, mostraremos ejemplos de implementaciones en \textit{Python} y discutiremos los resultados obtenidos.

\subsection{Algunas ecuaciones importantes}

Estas ecuaciones pueden describir fenómenos aleatorios en multiples áreas del conocimiento. Algunas aplicaciones incluyen modelos en física estadística, finanzas, ingeniería, química y biología.

\subsubsection{Ecuación del calor estocástica con ruido aditivo}

La ecuación del calor usual en dimensión 1 es una EDP de la siguiente forma:

\[
  \partial_t u(t,x) = \sigma\partial_{xx} u(t,x) + f(t,x), \quad t \in [0,T], x \in \mathcal{D}
\]

donde $f$ es un término fuente que puede depender del tiempo y del espacio, $\mathcal{D} \subseteq \mathbb{R}$ y $\sigma >0$ es un coeficiente de difusividad térmica. Esta ecuación describe, por ejemplo, la difusión de calor en un medio, también es ampliamente utilizada en finanzas para modelar la evolución de precios de activos.

Su versión estocástica con ruido aditivo es una ecuación de la siguiente forma:

\[
  \partial_t u(t,x) = \sigma\partial_{xx} u(t,x) + f(t,x) + \dot{W}(t,x), \quad t \in [0,T], x \in \mathcal{D}
\]

donde $\dot{W}(t,x)$ es un ruido blanco tiempo espacial, que representa una perturbación aleatoria en el sistema, que tiene media nula en cada punto espacio temporal y varianza unitaria, además, es independiente en el tiempo y en el espacio, esto puede representar, por ejeplo, una fuente de calor aleatoria que esta afectando al sistema.

Es importante notar que el término $\dot{W}(t,x)$ es un proceso estocástico y por lo tanto, esta asociado a un espacio de probabilidad subyacente $(\Omega, \mathcal{F}, \mathbb{P})$. Luego la ecuación anterior también se puede escribir como:

\[
  \partial_t u(t,x ;\omega) = \sigma\partial_{xx} u(t,x ; \omega) + f(t,x ; \omega) + \dot{W}(t,x ; \omega), \quad t \in [0,T], x \in \mathcal{D}, \omega \in \Omega
\]

Entonces la solución de esta ecuación es un proceso estocástico $u(t,x ; \omega)$ que depende del tiempo, del espacio y de la realización $\omega$.
